\begin{figure}
	\centering
		\includegraphics[scale=.45]{figs/The_proposed_model_workflow.jpeg}
	\caption{The proposed model workflow}
	\label{model_workflow}
\end{figure}


{\verb|hypothesis|}


\begin{enumerate}
\item[$\bullet$] Recipe A ingredients are: 300g chicken, 60g cheese, 100g potatoes
\item[$\bullet$] Recipe B ingredients are: 100g chicken, 120g cheese, 100g potatoes
\end{enumerate}


\begin{verbatim}
 \newtheorem{theorem}{Theorem}
 \newtheorem{lemma}[theorem]{Lemma}
 \newdefinition{rmk}{Remark}
 \newproof{pf}{Proof}
 \newproof{pot}{Proof of Theorem \ref{thm2}}
\end{verbatim}


\newtheorem{theorem}{Theorem}

\begin{theorem}
For system (8), consensus can be achieved with 
$\|T_{\omega z}$ ...
\begin{eqnarray}\label{10}
....
\end{eqnarray}
\end{theorem}


\begin{verbatim}
 \begin{enumerate}[Step 1.]
  \item This is the first step of the example list.
  \item Obviously this is the second step.
  \item The final step to wind up this example.
 \end{enumerate}
\end{verbatim}





{\verb|AG|}


$\mathbf{F}_{aug}$

$\mathbf{F}_{expand}$


\textsc{0SHOT-TC}



{Politics} &  \itshape{CNNPolitics Morning_Joe A side-by-side poll of what makes Obama the most popular politician in the US shows him as much closer to winning the Electoral College than he was a year ago. From CNN.} \\
{Politics}  & \itshape{The Trump era was about to turn into a nightmare. The administration had the right guy. The right guy was Steve Bannon. That is why I'm very glad he ended up in the White House.} \\


ESPN is making a difference this coming weekend—but I won't mention anything about it because we hate your sport. Last April on Real Soccer Talk, former San Diego Earthquakes player, Cristiano Ronaldo said




%%%%% tsd 
\begin{figure}
\includegraphics[width=\textwidth]{fig1.eps}
\caption{A figure caption is always placed below the illustration.
Please note that short captions are centered, while long ones are
justified by the macro package automatically.} \label{fig1}
\end{figure}

\begin{theorem}
This is a sample theorem. The run-in heading is set in bold, while
the following text appears in italics. Definitions, lemmas,
propositions, and corollaries are styled the same way.
\end{theorem}
%
% the environments 'definition', 'lemma', 'proposition', 'corollary',
% 'remark', and 'example' are defined in the LLNCS documentclass as well.
%
\begin{proof}
Proofs, examples, and remarks have the initial word in italics,
while the following text appears in normal font.
\end{proof}
For citations of references, we prefer the use of square brackets
and consecutive numbers. Citations using labels or the author/year
convention are also acceptable. The following bibliography provides
a sample reference list with entries for journal
articles~\cite{ref_article1}, an LNCS chapter~\cite{ref_lncs1}, a
book~\cite{ref_book1}, proceedings without editors~\cite{ref_proc1},
and a homepage~\cite{ref_url1}. Multiple citations are grouped
\cite{ref_article1,ref_lncs1,ref_book1},
\cite{ref_article1,ref_book1,ref_proc1,ref_url1}.